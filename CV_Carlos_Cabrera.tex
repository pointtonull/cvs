\documentclass[11pt,a4paper]{moderncv}

\usepackage [english]{babel}
% character encoding
\usepackage[utf8]{inputenc}
\usepackage [autostyle, english = american]{csquotes}
\MakeOuterQuote{"}

%% moderncv themes
\moderncvstyle{classic} % black
%\moderncvtheme[blue]{classic}
%\moderncvtheme[red]{classic}
%\moderncvtheme[grey]{classic}
%\moderncvtheme[grey]{casual}
%\moderncvstyle{casual}

% adjust the page margins
\usepackage[scale=.7]{geometry}
\recomputelengths

\firstname{Carlos M.} \familyname{Cabrera}
\title{Data Engineer}                                                            % title:de
\title{Python Expert}                                                            % title:pe
\title{Senior Software Engineer}                                                 % title:sse
\phone{+54-9387-611-0717}
\email{point.to@gmail.com}
%\social{http://pointtonull.github.io}
\social[linkedin]{pointtonull}
\social[github]{pointtonull}

\quote{Python engineer with 10+ years of experience, seeking roles in AI and ML.} % title:pe,sse
\quote{Python expert with experience in Data Mining and Machine Learning.}       % title:de

\emptysection{}

\nopagenumbers{}


\begin{document}

\hyphenation{Fa-cul-tad}

\maketitle

\cvitem{Note:}{\small This is the short version of this document.}                      % version:short
\cvitem{Note:}{\small This is the academic version of this document.}                   % version:academic

\section{Top Skills}

    \cvcomputer
        {Programming}{\textbf{Python, C, JavaScript, Go, Rust, Haskell, Erlang}} % title:de
        {Programming}{\textbf{Python, JavaScript, C, Rust}}                      % title:pe
        {Programming}{\textbf{Python, C, JavaScript, Go, Rust, Ruby, Haskell}}   % title:sse
        {Databases}{\textbf{DynamoDB, MongoDB, PostgreSQL, MySQL}}

    \cvcomputer
        {Cloud}{\textbf{AWS, Google}}
        {API}{\textbf{Chalice, Django, Flask, Tornado}}

    \cvcomputer
        {Scripting}{\textbf{AWK, Sh/Bash, Sed, Perl}}
        {Others}{\textbf{Scripting, Git, RESTfull, Linux}}                       % title:de
        {Others}{\textbf{Scripting, Git, RESTfull, Linux}}                       % title:pe
        {Others}{\textbf{Linux, Scripting, Git, RESTfull}}                       % title:sse


\section{Recent Professional Experience}                                         % version:short
\section{Professional Experience}                                                % version:long
\section{Professional Experience}                                                % version:academic

    \cventry
        {2018--Present}
        {Data Engineer}                                                          % title:de
        {Senior Software Engineer}                                               % title:pe,sse
        {\href{https://www.white.space}{White Space}} {} {}
        {
            Architect and Developer of a Data platform for Finantial
                Market Research.
            Cloud Architect using AWS tools.
            Impulsor of Serverless solutions and creator of CI/CD                % title:sse
                processes.                                                       % title:sse
            Maintainer of C++/Python embedded libraries for CPU
                intensive operations.
        }

    \cventry
        {2015--2017}
        {Data Engineer}                                                          % title:de
        {Senior Software Engineer}                                               % title:pe,sse
        {\href{https://alertlogic.com}{Alert Logic}} {} {}
        {
            Lead Programmer in the creation of self-adapting system for
                monitoring and reporting on security leaks.
            Creationg of Reputation algotirthm. Lead developer for
                Watchlist project.
            Main responsible for development of Universal Threats
                Database with Serverless architecture.
        }

    \cventry
        {2014--2015}
        {Data Analyst}
        {\href{https://alertlogic.com}{Alert Logic}} {} {}
        {
            Team Leader for team in charge of creation of parsing rules
                for high-performance natural language processor.
            Responsable of the creation of a Tooling effort and author 
                of the first version of the tooling framework.
        }

    \cventry
        {2012--2013}
        {Programmer -- Researcher}
        {\href{http://github.com/pointtonull/golsoft}{Laser Optics Group}} {} {}
        {
            Python Multiplatform System for acquisition and reconstruction of
                digital holograms that integrates multiple Machine Learning
                algorithms that automates the entire process.
        }

    \cventry
        {2010--2011}
        {Teacher}
        {National University of Salta} {} {}
        {
            Extension courses: RESCD-EXA: 741/2011,  RESCD-EXA:495/2010 and
                RESCD-EXA: 316/2011. GNU/Linux Debian Advanced Administration,
                System and Network administration, Introduction to scripting
                (Sh, Python, Perl, AWK) and services securing.
        }

    \cventry
        {2011}
        {Programmer}
        {\href{http://github.com/pointtonull/S60Salesman}{S60Salesman}} {} {}
        {
            Point of Sales System for Traveling Salesman written on Python for
                Symbian Smartphones.
        }

    \cventry
        {2011}
        {Development Leader}
        {\href{http://github.com/pointtonull/Fiscal3G}{Fiscal3G}} {} {}
        {
            System to control Massive Distributed teams. Can handle
                disconnection using SMS as secondary channel. Leader of team of
                3.
        }

    \cventry
        {2010--2011}
        {Systems Development Manager}
        {\href{http://ianux.com.ar}{IANUX Solutions}} {} {}
        {
            Planning and management of development projects. Survey, analysis
                and design of systems. Staff in charge 2. We have implemented
                solutions for different clients using Django and TurboGears.
        }

    \cventry
        {2009}
        {Programmer}
        {\href{http://tokuah.com.ar}{Tokuah}} {} {}
        {
            Design and implementation of an educational game using
                Python/Pygame.
            The project sought to instill values of respect                      % version:long
                toward the native populations.                                   % version:long
        }

    \cventry
        {2008--2009}
        {Analyst / Programmer -- Researcher}
        {\href{http://www.gendarmeria.gov.ar}{Argentina National Gendarmerie}} {} {}
        {
            Design and implementation of a complex radar system. Research and    % version:long
                design of DSP filters and analysis algorithms.                   % version:long
            Intensive use of Scientific Python Stack and Data Mining.            % version:long
        }

    \cventry
        {2008}
        {Analyst / Programmer}
        {COMPTI} {} {}
        {
            Design and implementation of a e-commerce System on top of           % version:long
            Sachmo/Django.                                                       % version:long
            Implementation of an API to price VoIP services over Python-SIP and  % version:long
            Asterisk.                                                            % version:long
        }

    \cventry
        {2007--2008}
        {Programmer}
        {\href{http://softwarelibre.unsa.edu.ar/dsa/}{Deliberative Council of Salta}}{Salta}{}
        {
            Developing applications for digital government. Adaptation and       % version:long
                documentation of "Request Tracker", a Perl Ticket Tracker        % version:long
                System.                                                          % version:long
        }

    \cventry
        {2006}
        {Programmer}
        {Bombito} {} {}
        {
            Design and implementation of a integrated system for Radio Air       % version:long
            Management.                                                          % version:long
            Implemented on Python and IMMS.                                      % version:long
        }


\section{Higher Education}

    \cventry
        {2008--2012}
        {System Analist}
        {"Dr. Facundo de Zuviría"}{Master}{}{}{}

    \subsection{Master Thesis}
        \cvline{Title}
            {\emph{Agile Development of Scientific Applications}}
        \cvline{Supervisor}
            {PhD. Andrea Carolina Monaldi}
        \cvline{Description}
            {
                \small Analyzes the use of agile methodologies in software
                    engineering for the field of scientific research.
                We made a suite for making and processing Digital Holograms on
                    top of the Python Scientific stack applying the Crystal
                    Clear agile techniques.
                    \url{http://github.com/pointtonull/golsoft}
            }


\section{Academic works} % version:academic,long

    \subsection{Publications}                                                    % version:academic,long

        \cventry                                                                 % version:academic,long
            {2012}                                                               % version:academic,long
            {                                                                    % version:academic,long
                Obtaining synthetic phase maps in digital holographic            % version:academic,long
                    microscopy using two wavelengths                             % version:academic,long
            }                                                                    % version:academic,long
            {Monaldi A C, Romero G G, Alanís E E, Cabrera C M}                   % version:academic,long
            {97 National Meeting of the Physical Association of Argentina}       % version:academic,long
            {2012}                                                               % version:academic,long
            {}                                                                   % version:academic,long

        \cventry                                                                 % version:academic,long
            {2012}                                                               % version:academic,long
            {                                                                    % version:academic,long
                Automatic compensation of phase aberration in digital            % version:academic,long
                    holographic microscopy off-axis configuration                % version:academic,long
            }                                                                    % version:academic,long
            {                                                                    % version:academic,long
                Monaldi A C, Romero G G, Cabrera C M                             % version:academic,long
            }                                                                    % version:academic,long
            {97 National Meeting of the Physical Association of Argentina}       % version:academic,long
            {2012}                                                               % version:academic,long
            {}                                                                   % version:academic,long

        \cventry                                                                 % version:academic,long
            {2012}                                                               % version:academic,long
            {                                                                    % version:academic,long
                Filter performance evaluation and implementation of an           % version:academic,long
                autofocus method in reconstruction of holograms in digital       % version:academic,long
                holographic microscopy                                           % version:academic,long
            }                                                                    % version:academic,long
            {Cabrera C M, Monaldi A C, Romero G G}                               % version:academic,long
            {97 National Meeting of the Physical Association of Argentina}       % version:academic,long
            {2012}                                                               % version:academic,long
            {}                                                                   % version:academic,long


\subsection{Named Lectures}                                                      % version:academic

\cventry                                                                         % version:academic
    {2010}                                                                       % version:academic
    {`Programming is fun'}                                                       % version:academic
    {Fifth Conferences of Free Software of Salta}                                % version:academic
    {National University of Salta}                                               % version:academic
    {}                                                                           % version:academic
    {                                                                            % version:academic
        About the importance of good programming practices.                      % version:academic
        Introducing Pomodoro and other time management techniques.               % version:academic
        The value of frameworks.                                                 % version:academic
    }                                                                            % version:academic

\cventry                                                                         % version:academic
    {2010}                                                                       % version:academic
    {`The Development in the Free Software World'}                               % version:academic
    {Software Freedom Day}                                                       % version:academic
    {Catholic University of Salta}                                               % version:academic
    {}                                                                           % version:academic
    {                                                                            % version:academic
        Introductory talk and motivational directed to beginning programmers.    % version:academic
    }                                                                            % version:academic

\cventry                                                                         % version:academic
    {2009}                                                                       % version:academic
    {`from 0 to Python in 30 minutes'}                                           % version:academic
    {Latin American Free Software InstallFest}                                   % version:academic
    {National University of Salta}                                               % version:academic
    {}                                                                           % version:academic
    {                                                                            % version:academic
        Brief presentation of philosophy and syntax of Python                    % version:academic
        to demonstrate it takes very few time to start                           % version:academic
        coding in Python.                                                        % version:academic
    }                                                                            % version:academic

\cventry                                                                         % version:academic
    {2008}                                                                       % version:academic
    {`from 0 to Python in 45 minutes'}                                           % version:academic
    {Latin American Free Software InstallFest}                                   % version:academic
    {National University of Salta}                                               % version:academic
    {}                                                                           % version:academic
    {                                                                            % version:academic
        Brief presentation of philosophy and syntax of Python                    % version:academic
        to demonstrate it takes very few time to start                           % version:academic
        coding in Python.                                                        % version:academic
        Include a general description of paradigms in Python.                    % version:academic
    }                                                                            % version:academic

\cventry                                                                         % version:academic
    {2007}                                                                       % version:academic
    {Disertación sobre `Distribuciones GNU/Linux'}                               % version:academic
    {Festival Latinoamericano de Software Libre}                                 % version:academic
    {Universidad Nacional de Salta}                                              % version:academic
    {}                                                                           % version:academic
    {}                                                                           % version:academic

\cventry                                                                         % version:academic
    {2007}                                                                       % version:academic
    {Series of talks: `Object Oriented Programming in Python'}                   % version:academic
    {Facultad de Ciencias Exactas de la Universidad Nacional de Salta}           % version:academic
    {}                                                                           % version:academic
    {}                                                                           % version:academic
    {                                                                            % version:academic
        Topics:                                                                  % version:academic
        concepts of objects,                                                     % version:academic
        interpreter architecture,                                                % version:academic
        first-class objects,                                                     % version:academic
        functions as objects (callables),                                        % version:academic
        classes declaration syntax,                                              % version:academic
        inheritance operations,                                                  % version:academic
        performance and style considerations.                                    % version:academic
    }                                                                            % version:academic

    \subsection{Conferences Organized or Co-Organized}                           % version:academic

        \cventry                                                                 % version:academic
            {2011}                                                               % version:academic
            {Regional Coordinator}                                               % version:academic
            {National University of Salta}                                       % version:academic
            {}                                                                   % version:academic
            {Latin American Free Software InstallFest}                           % version:academic
            {}                                                                   % version:academic

        \cventry                                                                 % version:academic
            {2007--2011}                                                         % version:academic
            {Organization}                                                       % version:academic
            {National University of Salta}                                       % version:academic
            {}                                                                   % version:academic
            {Annual Conferences of Free Software of Salta.}                      % version:academic
            {}                                                                   % version:academic

        \cventry                                                                 % version:academic
            {2010}                                                               % version:academic
            {Organization}                                                       % version:academic
            {Catholic University of Salta}                                       % version:academic
            {}                                                                   % version:academic
            {Software Freedom Day.}                                              % version:academic
            {}                                                                   % version:academic

        \cventry                                                                 % version:academic
            {2008--2010}                                                         % version:academic
            {Organization}                                                       % version:academic
            {National University of Salta}                                       % version:academic
            {}                                                                   % version:academic
            {Latin American Free Software InstallFest.}                          % version:academic
            {}                                                                   % version:academic

\section{Selected Open Source Projects}

%    % FSD
%    \cventry{}{\href{https://github.com/pointtonull/browser}{BROWSER}} % title:fsd
%        {Python Library for statefull browser emulation, used for web-scrapping and security research.}{}{}{} % title:fsd
%    \cventry{}{\href{https://github.com/pointtonull/golsoft}{GOLSoft}} % title:fsd
%        {Digital Holography Processing Framework designed to train Deep Learning algorithms.}{}{}{} % title:fsd
%    \cventry{}{\href{https://github.com/pointtonull/Fisgon}{Fisgon}} % title:fsd
%        {Security PoC to show how easy is to exploit low-severity credentials leaks when the users reuse their passwords.}{}{}{} % title:fsd

    % SSE

    \cventry{}                                                                         % title:sse
        {\href{https://github.com/pointtonull/golsoft}{GOLSoft}}                       % title:sse
        {Digital Holography Framework designed                                         % title:sse
         to train Deep Learning algorithms}                                            % title:sse
        {}{}{}                                                                         % title:sse

    \cventry{}                                                                         % title:sse
        {\href{https://github.com/pointtonull/vimrenamer}{VimRenamer}}                 % title:sse
        {Vim extension to bulk-edit tons of files with some few key-strokes}           % title:sse
        {}{}{}                                                                         % title:sse

    \cventry{}                                                                         % title:sse
        {\href{https://github.com/pointtonull/browser}{BROWSER}}                       % title:sse
        {Python Library for statefull browser emulation,                               % title:sse
         used for web-scrapping and security research}                                 % title:sse
        {}{}{}                                                                         % title:sse

    % DE

    \cventry{}                                                                          % title:de
        {\href{https://github.com/pointtonull/golsoft}{GOLSoft}}                        % title:de
        {Digital Holography Framework designed                                          % title:de
         to train Deep Learning algorithms}                                             % title:de
        {}{}{}                                                                          % title:de

    \cventry{}                                                                          % title:de
        {\href{https://github.com/pointtonull/vimrenamer}{VimRenamer}}                  % title:de
        {Vim extension to bulk-edit tons of files with some few key-strokes}            % title:de
        {}{}{}                                                                          % title:de

    \cventry{}                                                                          % title:de
        {\href{https://github.com/pointtonull/pyunwrap}{PyUnwrap}}                      % title:de
        {A efficient Ansi C wave unwrapper with python bindings}                        % title:de
        {}{}{}                                                                          % title:de

    % PE

    \cventry{}                                                                          % title:pe
        {\href{https://github.com/pointtonull/golsoft}{GOLSoft}}                        % title:pe
        {Digital Holography Framework designed                                          % title:pe
         to train Deep Learning algorithms}                                             % title:pe
        {}{}{}                                                                          % title:pe

    \cventry{}                                                                          % title:pe
        {\href{https://github.com/pointtonull/vimrenamer}{VimRenamer}}                  % title:pe
        {Vim extension to bulk-edit tons of files with some few key-strokes}            % title:pe
        {}{}{}                                                                          % title:pe

    \cventry{}                                                                          % title:pe
        {\href{https://github.com/pointtonull/pyunwrap}{PyUnwrap}}                      % title:pe
        {A efficient Ansi C wave unwrapper with python bindings}                        % title:pe
        {}{}{}                                                                          % title:pe

\section{Complementary Formation}
%\subsection{Systems Analyst}
% \cventry{year--year}{Degree}{Institution}{City}{\textit{Grade}}{Description}  % arguments 3 to 6 are optional

\cventry
    {2018}
    {Advanced Styling with Responsive Design}
    {License
        \href{https://www.coursera.org/account/accomplishments/records/ELDM5LRGTK3H}
        {ELDM5LRGTK3H}
    }
    {Michigan University}
    {\textit{Score 92.3\%}}
    {
        Topics include:
            Fluid Measurements,
            Pixel to Em,
            Dynamic change,
            relative and absolute,
            Media Queries,
            Fluid Measurements and Media Queries,
            Wire Frames,
            Breakpoints,
            Responsive Navigation,
            Media Queries and breakpoints,
            Twitter Bootstrap 4 (
                Breakpoints,
                Grid System,
                Navigation,
                Standards vs Convenience,
                Responsive Images,
                Tables,
                Advanced Navigation
                )
    }

\cventry
    {2018}
    {Structuring Machine Learning Projects}
    {License
        \href{https://www.coursera.org/account/accomplishments/records/6QNQ9XWER98A}
        {6QNQ9XWER98A}
    }
    {Michigan University}
    {\textit{Score 86.7\%}}
    {
        Topics include:
            ML Strategy,
            Comparing to human-level performance,
            Carrying out error analysis,
            Cleaning up incorrectly labeled data,
            Iterative improvement,
            Mismatched training and dev/test set,
            Training and testing on different distributions,
            Bias and Variance with mismatched data distributions,
            Addressing data mismatch,
            Learning from multiple tasks,
            Transfer learning,
            Multi-task learning,
            End-to-end deep learning,
            Case: Flight simulator,
            Case: Bird recognition,
            Case: Autonomous driving.
    }

\cventry
    {2018}
    {Interactivity with JavaScript}
    {License
        \href{https://www.coursera.org/account/accomplishments/records/LZ74JEGGWGS2}
        {LZ74JEGGWGS2}
    }
    {University of Michigan}
    {\textit{Score 98.7\%}}
    {
        Topics include:
            DOM,
            DOM with OOP,
            Output,
            Variables,
            Data Types,
            Operators and Expressions,
            CodePen,
            Debugging,
            Functions,
            Code Placement and organization,
            Events and Functions,
            "this",
            Arrays and Looping,
            Advanced Coding Techniques,
            JavaScript Iteration,
            Flow Of Control,
            Advanced Conditionals,
            Common Errors,
            Validating Form Data,
            Forms (
                Validation,
                Checkboxes and Radio Buttons,
                Using Forms on Your Site
                ),
            JQuery,
            Autocomplete with JavaScript.
    }

\cventry
    {2018}
    {Improving Deep Neural Networks: Hyperparameter tuning, Regularization and Optimization}
    {License
        \href{https://www.coursera.org/account/accomplishments/records/X932C9K84YAK}
        {X932C9K84YAK}
    }
    {deeplearning.ai}
    {\textit{Score 96.1\%}}
    {
        Topics include:
            Train / Dev / Test sets
            Bias / Variance
            Basic Recipe for ML
            Regularization
            Why? (
                Dropout,
                Other methods
                )
            Normalizing inputs
            Vanishing / Exploding gradients
            Weight Initialization
            Numerical approximation of gradients
            Gradient checking
            Initialization
            Regularization
            Optimization algorithms
            Mini-batch gradient descent
            Exponentially weighted averages
            Bias correction
            Gradient descent with momentum
            RMSprop
            Adam optimization algorithm
            Learning rate decay
            The problem of local optima
            Optimization algorithms
            Hyperparameter tuning, Batch Normalization and Programming Frameworks
            Tuning process
            Using an appropriate scale to pick hyperparameters
            Pandas vs. Caviar
            Normalizing activations in a network
            Fitting Batch Norm into a neural network
            Batch Norm at test time
            Softmax Regression
            Training a softmax classifier
            TensorFlow
    }

\cventry
    {2018}
    {Learning GraphQL}
    {LI Learning}
    {LinkedIn}
    {\textit{Score 100\%}}
    {
        Topics include:
            GraphiQL and the GitHub API,
            GraphQL Queries,
            Understanding Schemas,
            Query the \_\_schema,
            Handling Data (
                Aliases,
                Fragments,
                Nested fields,
                Connections,
                Pagination
            ),
            Operations and Variables (
                Operation names,
                variable definitions,
                mutations
            ).
    }

\cventry
    {2018}
    {Machine Learning \& AI Foundations: Value Estimations}
    {LI Learning}
    {LinkedIn}
    {\textit{Score 100\%}}
    {
        Topics include:
            Supervised machine learning for value prediction,
            Find the best weights automatically,
            An Overview of Building a Machine Learning System,
            Introduction to NumPy, scikit-learn, and pandas,
            Gradient boosting,
            Training Data,
            Feature engineering,
            Feature selection,
            Coding Our System,
            Measure accuracy,
            Overfitting and underfitting.
    }

\cventry
    {2018}
    {Python for Data Science Essential Training}
    {LI Learning}
    {LinkedIn}
    {\textit{Score 100\%}}
    {
        Topics include:
            Jupyter,
            Data Munging,
            Data Visualization,
            Math and Statistics,
            NumPy arithmetic,
            Dimensionality Reduction,
            Explanatory factor analysis,
            Principal component analysis (PCA),
            Outlier Analysis,
            Extreme value analysis using univariate methods,
            Multivariate analysis for outlier detection,
            Cluster Analysis,
            Network Analysis with NetworkX,
            Basic Algorithmic Learning,
            Web-based Data Visualizations with Plotly,
            Web Scraping with Beautiful Soup.
    }

\cventry
    {2018}
    {Introduction to HTML5}
    {License
        \href{https://www.coursera.org/account/accomplishments/records/LZ74JEGGWGS2}
        {LZ74JEGGWGS2}
    }
    {University of Michigan}
    {\textit{Score 95.7\%}}
    {
        Topics include:
            The Evolution of HTML,
            Page Requests,
            Browsers,
            Editors,
            The Document Object Model [DOM],
            HTML5 Tags and Syntax,
            Semantic Tags,
            Template Page,
            Images,
            Hyperlinks,
            Multimedia,
            Tables,
            Useful Tags,
            Accessibility,
            Validation,
            Hosting,
            cPanel,
            Using SFTP.
    }

\cventry
    {2018}
    {Amazon Web Services: Monitoring and Metrics}
    {LI Learning}
    {LinkedIn}
    {\textit{Score 100\%}}
    {
        Topics include:
            Monitoring Tools,
            Understand CloudWatch,
            Extend CloudWatch,
            Explore AWS Config,
            Add Managed Config Rules,
            Beyond Traditional Monitoring,
            Understand AWS logging,
            Understand Elasticsearch,
            Application performance management,
            Link CloudWatch and Lambda.
    }

\cventry
    {2018}
    {Introduction to CSS3}
    {License
        \href{https://www.coursera.org/account/accomplishments/records/NDM4GGJPEETK}
        {NDM4GGJPEETK}
    }
    {University of Michigan}
    {\textit{Score 97.9\%}}
    {
        Topics include:
            Cascading Style Sheets,
            Display and Visibility,
            Styling Syntax and Theory,
            Box Model,
            Advanced Selectors,
            Shorthand rules,
            Browser Capabilites,
            Headings,
            Homework Two Description,
            Advanced Style (
                Psuedo-classes,
                Pseudo-elements,
                Transitions,
                and Positioning
            ),
            Transforms,
            Animation,
            Tables,
            Navigation Menus,
            Accessible Navigation,
            Accessibility of Headings,
    }

\cventry
    {2018}
    {Learning Cloud Computing: Monitoring and Operations}
    {LI Learning}
    {LinkedIn}
    {\textit{Score 100\%}}
    {
        Topics include:
            Cloud health monitoring,
            Cloud performance monitoring,
            Cloud security monitoring,
            Cloud governance monitoring,
            Short-term cloud monitoring analytics,
            Long-term cloud monitoring analytics,
            Cloud cost monitoring,
            AWS CloudWatch,
            Datadog,
            Librato CloudWatch,
            Cloud Cruiser,
            Microsoft cloud monitoring,
            Rackspace cloud monitoring,
            Other players,
            Understanding requirements,
            Creating plan,
            Selecting tools.
    }

\cventry
    {2018}
    {Neural Networks and Deep Learning}
    {License
        \href{https://www.coursera.org/account/accomplishments/records/T7YRLSDSU6A}
        {3T7YRLSDSU6A3}
    }
    {deeplearning.ai}
    {\textit{Score 100\%}}
    {
        Topics include:
            Introduction to deep learning,
            Supervised Learning with Neural Networks,
            Neural Networks Basics,
            Binary Classification,
            Logistic Regression,
            Logistic Regression Cost Function,
            Gradient Descent,
            Derivatives,
            Computation graph,
            Derivatives with a Computation Graph,
            Logistic Regression Gradient Descent,
            Vectorization,
            Broadcasting in Python,
            Jupyter/iPython Notebooks,
            Logistic Regression with a Neural Network mindset,
            Shallow neural networks,
            Neural Network Representation,
            Computing a Neural Network's Output,
            Vectorizing across multiple examples,
            Explanation for Vectorized Implementation,
            Activation functions,
            Derivatives of activation functions,
            Gradient descent for Neural Networks,
            Backpropagation intuition,
            Random Initialization,
            Planar data classification with a hidden layer,
            Shallow Neural Networks,
            Deep Neural Networks,
            Deep L-layer neural network,
            Forward Propagation,
            Deep representations?,
            Building blocks of deep neural networks,
            Forward and Backward Propagation,
            Parameters vs Hyperparameters,
            Applications.
    }

\cventry
    {2017}
    {JavaScript Essential}
    {LI Learning}
    {LinkedIn}
    {\textit{Score 100\%}}
    {
        Topics include:
            Tools for JS Development,
            Working with data,
            Arrays,
            Functions and Objects,
            Variable scope,
            ES2015,
            Object constructors,
            DOM Manipulation,
            querySelector methods,
            Use CSS with JS,
            Events,
            Debug JS,
            Linting,
            Minification.
    }

\cventry
    {2017}
    {Building Deep Learning Applications with Keras 2.0}
    {LI Learning}
    {LinkedIn}
    {\textit{Score 100\%}}
    {
        Topics include:                                                          % version:long,academic
            Keras Overview,                                                      % version:long,academic
            Setting Up,                                                          % version:long,academic
            Creating a Neural Network in Keras,                                  % version:long,academic
            Training Models,                                                     % version:long,academic
            Pre-Trained Models in Keras,                                         % version:long,academic
            Monitoring a Keras model with TensorBoardd,                          % version:long,academic
            Using a trained Keras Model in Google Cloud.                         % version:long,academic
    }

\cventry
    {2016}
    {CSSLP: Secure Software Implementation and Coding}
    {Certified Secure Software Lifecycle Professinoal}
    {Skillsoft}
    {\textit{Score 100\%}}
    {
        Topics include:                                                          % version:long,academic
            Declarative versus programmatic security,                            % version:long,academic
            OWASP and CWE,                                                       % version:long,academic
            and some defense coding practices and controls,                      % version:long,academic
            error handling,                                                      % version:long,academic
            and session management.                                              % version:long,academic
            Essential secure coding techniques (                                 % version:long,academic
                versioning,                                                      % version:long,academic
                peer-based code reviews,                                         % version:long,academic
                code analysis,                                                   % version:long,academic
                and anti-tampering techniques                                    % version:long,academic
            ).                                                                   % version:long,academic
    }

\cventry
    {2016}
    {AWS Business Professional}
    {Training and Certification}
    {amazon.com}
    {\textit{Score 100\%}}
    {
        Topics include:                                                          % version:long,academic
            The AWS Business Professional training program provides              % version:long,academic
                participants with basic knowledge of AWS products                % version:long,academic
                and services.                                                    % version:long,academic
            These online and instructor-led training modules help                % version:long,academic
                build the foundation to effectively leverage AWS                 % version:long,academic
                solutions to customers.                                          % version:long,academic
            This course is intended for individuals responsible for              % version:long,academic
                articulating the business benefits of AWS services               % version:long,academic
                and how AWS solutions can help solve common business             % version:long,academic
                problems.                                                        % version:long,academic
    }

\cventry
    {2012}
    {Neural Networks for Machine Learning}
    {MOOC by Prof. Geoffrey Hinton}
    {Toronto University}
    {}
    {
        Topics include:                                                          % version:long
            The Perceptron learning procedure.                                   % version:long
            The backpropagation.                                                 % version:long
            Learning feature vectors for words.                                  % version:long
            Object recognition with neural nets.                                 % version:long
            Optimization.                                                        % version:long
            Recurrent neural networks.                                           % version:long
            Improving generalization.                                            % version:long
            Combining multiple neural networks to improve generalization.        % version:long
            Hopfield nets and Boltzmann machines.                                % version:long
            Restricted Boltzmann machines (RBMs).                                % version:long
            Deep Belief Nets.                                                    % version:long
            Deep neural nets with generative pre-training.                       % version:long
            Modeling hierarchical structure with neural nets.                    % version:long
    }

\cventry
    {2012}
    {Social Network Analysis}
    {MOOC by Prof. Lada Adamic}
    {Michigan University}
    {}
    {
        Topics include:                                                          % version:long
            Networks,                                                            % version:long
            Crowds and Markets.                                                  % version:long
            Erdos-Renyi and Barabasi-Albert.                                     % version:long
            Network centrality.                                                  % version:long
            Community structure.                                                 % version:long
            Small world network models,                                          % version:long
            optimization,                                                        % version:long
            strategic network formation and search.                              % version:long
            Contagion,                                                           % version:long
            opinion formation,                                                   % version:long
            coordination and cooperation.                                        % version:long
            Cool and unusual applications of SNA@.                               % version:long
            SNA and social media.                                                % version:long
            Network resilience.                                                  % version:long
    }

\cventry
    {2012}
    {Machine Learning}
    {MOOC by Prof. Andrew Ng}
    {Stanford University}
    {\textit{Score 100\%}}
    {
        Topics include:                                                          % version:long
            Supervised learning (                                                % version:long
                    parametric/non-parametric algorithms,                        % version:long
                    support vector machines,                                     % version:long
                    kernels,                                                     % version:long
                    neural networks                                              % version:long
                ).                                                               % version:long
            Unsupervised learning (                                              % version:long
                    clustering,                                                  % version:long
                    dimensionality reduction,                                    % version:long
                    recommender systems,                                         % version:long
                    deep learning                                                % version:long
                ).                                                               % version:long
            Best practices (                                                     % version:long
                    bias/variance theory;                                        % version:long
                    innovation process in machine learning and AI                % version:long
                ).                                                               % version:long
                text understanding (                                             % version:long
                    web search,                                                  % version:long
                    anti-spam                                                    % version:long
                ),                                                               % version:long
                computer vision,                                                 % version:long
                medical informatics,                                             % version:long
                audio,                                                           % version:long
                database mining,                                                 % version:long
                and other areas.                                                 % version:long
    }

\cventry
    {2012}
    {Model Thinking}
    {MOOC by Prof. Scott E. Page - Ann Arbor}
    {Michigan University}
    {
        \textit{Score 100\%} % version:long
    }
    {
        Topics include:                                                          % version:long
            Peer Effects.                                                        % version:long
            Decision Theory,                                                     % version:long
            Multi-Criterion/Spatial Models,                                      % version:long
            Decision Trees,                                                      % version:long
            The Value of Information.                                            % version:long
            Behavioral Economics.                                                % version:long
            Linear Models,                                                       % version:long
            Nonlinear Data.                                                      % version:long
            Tipping Points,                                                      % version:long
            Economic Growth.                                                     % version:long
            Diverse Perspectives and Heuristics of Groups.                       % version:long
            Markov Chains,                                                       % version:long
            Markov Convergence Theorem.                                          % version:long
            Efficient Market Hypothesis.                                         % version:long
            Collective Action Models.                                            % version:long
            Overcome Problems of Hidden Action and Hidden Information.           % version:long
            Learning,                                                            % version:long
            Replicator Dynamics.                                                 % version:long
            Diversity and Complexity,                                            % version:long
            Diversity and Prediction,                                            % version:long
       Collective Prediction.                                                    % version:long
    }

%\cventry{2004}{Computer Technician}{`Dr. Facundo de Zuviría'}{Salta}{\textit{Promedio: 7.55}}{}
%\cventry{2001}{e-Business, Security, Wireless and Ututo Linux}{`María del Rosario de San Nicolás'}{Salta}{}{}
%\cventry{1997}{Programación}{Instituto Argentino de Computación}{Salta}{\textit{Promedio: 10.00}}{}


%	\cventry{year--year}{Job title}{Employer}{City}{}{Description}  % arguments 3 to 6 are optional



%\section{Skills}
%	\cvlistitem{Highly analytical thinking with demonstrated talent for identifying, scrutinizing, improving, and streamlining complex work processes.}
%\cvlistitem{Goal-driven leader who maintains a productive climate and confidently motivates, mobilizes, and coaches employees to meet high performance standards.}
%	\cvlistitem{Personable professional whose strengths include cultural sensitivity and an ability to build rapport with a diverse workforce in multicultural settings.}


%\section{Values}
%	\cvlistitem{Results-driven achiever with exemplary planning and organizational skills, along with a high degree of detail orientation.}
%	\cvlistitem{Persistent and driven; acquired BA degree while working a fulltime job.}
%	\cvlistitem{Innovative problem-solver who can generate workable solutions and resolve complaints.}
%	\cvlistitem{Enthusiastic, knowledge-hungry learner, eager to meet challenges and quickly assimilate new concepts.}
%	\cvlistitem{Highly motivated self-starter who takes initiative with minimal supervision.}
%	\cvlistitem{Conscientious go-getter who is highly organized, dedicated, and committed to professionalism.}
%	\cvlistitem{Seasoned professional whose honesty and integrity provide for effective leadership and optimal business relationships.}


%\section{Languages}
%% \cvlanguage{language 1}{Skill level}{Comment}
%\cvlanguage{English}{Reading\hspace{4ex}(High) \\
%Writing\hspace{2.6ex}(High) \\
%Speak\hspace{5.6ex}(Low)}{Translator of the \href{`http://www.esdebian.org/dsa'}{`Debian Security Announces'} and another Technical Documents.}


%\section{Notes}
%%% % \cvline{hobby 1}{\small Descripción}
%	\cvline{Resume}{This is not an exhaustive enumeration; have been included only the most relevant items.}
%	\cvline{References}{References are not included as attention to the privacy of my former employers, these will be made available on request.}

%\closesection{}                   % needed to renewcommands
%\renewcommand{\listitemsymbol}{-} % change the symbol for lists


%Publications from a BibTeX file
%\nocite{*}
%\bibliographystyle{plain}
%\bibliography{publications}       % 'publications' is the name of a BibTeX file

%\section{Extra 2}
%\cvlistdoubleitem[\Neutral]{Item 3}{}
%\cvlistdoubleitem[\Neutral]{Item 1}{Item 4} % title:sse
%\cvlistdoubleitem[\Neutral]{Item 2}{Item 5} % title:pe
%\cvlistdoubleitem[\Neutral]{Item 2}{Item 5} % title:fsd

\end{document}

%% end of file `template_en.tex'.
